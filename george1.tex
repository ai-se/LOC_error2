
   To conduct that perturbation study, we 
   

   # george. please copy over our notes on the data sets, the perturbation study and the scott-knot method.
   # ensure your comments do the setup (with the data) . cpvers the stats methods, then discusses and explains all the results. note you need to end with the ``Results1'' bubble from p1.

     
\[\mathit{KLOC} = \mathit{KLOC}*((1- n) + (2*n*r))\]
where $n \in \{0.25,0.5\}$ is the level of noise we are exploring and $r$ is a random number
$0 \le r \le 1$. In \fig{noise}, any result
marked with {\em n/2} or {\em n/4} shows what happens
when the KLOCs were varied by 50\% or 25\% respectively.
In only one case (COC81) were the noisy results statistically
different from using data without noise. That is,
the parametric estimation method being recommended here is
not unduly effected by noise where the KLOC values
vary up to 50\% of their original value.
